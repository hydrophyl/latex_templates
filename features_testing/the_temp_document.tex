\documentclass[draft=false
              ,paper=a4
              ,twoside=false
              ,fontsize=12pt
              ,headsepline
              ,BCOR10mm
              ,DIV11
              ]{scrbook}
%%\usepackage[backend=biber]{biblatex}
%%\addbibresource{citation.bib}
\usepackage[ngerman,english]{babel}
\usepackage[T1]{fontenc}
\usepackage[utf8]{inputenc}
\usepackage[latin1]{inputenc}
\usepackage[top=2cm,left=2cm,right=2cm,bottom=2cm]{geometry}
\usepackage[parfill]{parskip} 
\usepackage{hyperref}
%% From HAW-Template
\usepackage{libertine}
\usepackage{pifont}
\usepackage{microtype}
\usepackage{textcomp}
\usepackage[german,refpage]{nomencl}
\usepackage{setspace}
\usepackage{makeidx}
\usepackage{listings}
\usepackage{natbib}
\usepackage[ngerman,colorlinks=true]{hyperref}
\usepackage{soul}
\usepackage{graphicx} %% Important for including pictures
\usepackage{lipsum} %% for sample text
\usepackage{fancyhdr} %% for aligning header/footer
\pagestyle{fancy}
\hypersetup{
  colorlinks=true,
  linkcolor=black, 
  filecolor=cyan,
  urlcolor=magenta,
  citecolor=red,
}
\linespread{1.25} %% Paragraphs 1.5 lineheight

\begin{document}

\frontmatter

%% output title page
\thispagestyle{empty}
\phantom{o}
\vspace{3cm}
\begin{center}
Term paper

\vspace{1em}

\textbf{\Large{National Action Plans to implement the UN Guiding Principles on
Business and Human Rights}}

\vspace{1cm}

HAW Hamburg

Hamburg University of Applied Sciences

Faculty of Business \& Social Sciences

Department of Business

\vspace{1cm}

Foreign Trade/ International Management, B.Cs.

Academic research and Writing, Summer term 2020

Professor Dr. Christian Decker

Instructor Professor Michael Gille
\end{center}

\vfill

\begin{table}[h]
\begin{tabular}{ll}
Matriculation number:    & 2507884                                                                    \\ \\[-1em]
Name:                    & Tran, Thy                                                                  \\ \\[-1em]
Date and place of birth: & 24 January 1998, Vietnam                                                   \\ \\[-1em]
Telephone:               & 015736506729                                                               \\ \\[-1em]
Address:                 & Emil-Andressen-Straße 34 B, 22529 Hamburg                                  \\ \\[-1em]
Email:                   & Thy.Tran@haw-hamburg.de                                                    \\ \\[-1em]
                         &                                                                            \\ \\[-1em]
Date of submission:      & 
\end{tabular}
\end{table}


\tableofcontents

\mainmatter
\chapter{Introduction}
\section{Research questions}

Since the end of 2019, the world has had faced with the biggest threat to the maintenance of international peace, security and economic stability. The coronavirus (also being called Covid-19) pandemic caused by severe acute respiratory syndrome coronavirus 2 (SARS-CoV-2). The outbreak was identified in Wuhan, China in December 2019 and officially recognized as a pandemic on 11 March 2020. On 12 April 2020, more than 1.84 million cases of Covid-19 have been reported in 210 countries and territories worldwide, more than 113 000 deaths and more than 421 000 people have recovered. This coronavirus spreads primarily through droplets generated when an infected person coughs or sneezes, or through droplets of saliva or discharge from the nose. 

To hinder the fast contagiousness of this coronavirus, many countries worldwide have responded by implementing travel restrictions, quarantines, curfews, workplace hazard controls and facility closures. This coronavirus outbreak has brought considerable human suffering and economic disruption, including the direct disruption to global supply chains, weaker final demand for imported goods and services, and the wider regional declines in international tourism and business travel. 

Because of the unpredictable goings-on of this outbreak and its negative effects on tourism sector, the international as well as national tourist organisations must have urgent response in time in order to mitigate this global crisis. Consequently, having crisis communication plans and strategies is undoubtedly necessary. Nonetheless, tourist organisations cannot implement the appropriated crisis communication strategies and reactions against the coronavirus without collaborating with other international organisations and national governments around the world.

This term paper will represent the influences of Covid-19 pandemic on tourism industry. The focus will be on analysing the crisis communication strategies and tools of international organisations, especially international tourist organisations in term of mitigating the severe global coronavirus crisis (and pointing out differentiations of crisis communication plans in Vietnam, one of the best countries avoiding and precluding the spread of coronavirus and the United States of America, the country has the world’s highest number of deaths and cases. )

\section{Course of investigation}

Based upon the research question postulated in chapter 1.1, the negative impacts of coronavirus on tourism industry will be described in chapter 2. Following the second chapter is the analyse of crisis communication strategies and tools of tourist organisations in term of dealing with the coronavirus outbreak. (The last chapter will describe the differences of crisis communication plans between Vietnam and The United States of America (USA) when facing coronavirus pandemic.)

\chapter{Influences of coronavirus pandemic on tourism industry}

\chapter{Crisis communication strategies of international tourism organisations}

\section{Implemented crisis communication strategies}
\subsection{Collaboration with worldwide organisations}
\subsection{Crisis communication tools and channels}

\section{The impacts of crisis communication strategies on the spread of coronavirus pandemic}

\chapter{Comparison crisis communication in Vietnam and in The United States of America in tourism industry}

\end{document}
