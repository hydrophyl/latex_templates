\documentclass[draft=false
              ,paper=a4
              ,twoside=false
              ,fontsize=12pt
              ,headsepline
              ,BCOR10mm
              ,DIV11
              ]{scrbook}
\usepackage[backend=biber,sorting=nty]{biblatex}
\addbibresource{citation.bib}
\usepackage[ngerman,english]{babel}
\usepackage[T1]{fontenc}
\usepackage[utf8]{inputenc}
\usepackage[top=2.5cm,left=3cm,right=2cm,bottom=2cm]{geometry}
\usepackage[parfill]{parskip} 
\usepackage{hyperref}
\usepackage{glossaries} %% to do abbreviations
%% From HAW-Template
\usepackage{setspace}
\usepackage{listings}
\usepackage{graphicx} %% Important for including pictures
\usepackage{fancyhdr} %% for aligning header/footer
\pagestyle{fancy}
\hypersetup{
  colorlinks=true,
  linkcolor=black, 
  filecolor=cyan,
  urlcolor=magenta,
  citecolor=red,
}
\linespread{1.25} %% Paragraphs 1.5 lineheight
\setlength{\parskip}{3pt}
%% \setmainfont{Arial}
\usepackage{uarial}
\renewcommand{\familydefault}{\sfdefault}

%% to set size of headings as normalsize 12pt
\usepackage{sectsty}
\chapterfont{\normalsize}
\sectionfont{\normalsize}
\subsectionfont{\normalsize}

%% to change space of headings to 0
\usepackage{titlesec}
\titlespacing*{\section}
{0pt}{0ex}{0ex}
\titlespacing*{\subsection}
{0pt}{0ex}{0ex}
\titlespacing*{\subsubsection}
{0pt}{0ex}{0ex}
\titleformat{\chapter}
  {\normalfont\normalsize\bfseries}{\thechapter}{1em}{}
\titlespacing*{\chapter}
{0pt}{0ex}{0ex}

%% Change title of \tableofcontents
\addto\captionsenglish{% Replace "english" with the language you use
  \renewcommand{\contentsname}%
    {I. OUTLINE}%
  \renewcommand{\listfigurename}%
  	{II. List of figures}
  \renewcommand{\listtablename}%
	{III. List of tables}
  \renewcommand{\refname}%
  	{V. List of references}
}

%% Generate the glossaries
\makeglossaries
\begin{document}
\newglossaryentry{COVID}{name=COVID-19, description={Coronavirus disease 2019}}
\newglossaryentry{who}{name=WHO, description={Word Health Organization}}
\newglossaryentry{unwto}{name=UNWTO, description={World Tourism Organization}}
\newglossaryentry{iata}{name=IATA, description={International Air Transport Association}}

\frontmatter

%% output title page
\thispagestyle{empty}
\phantom{o}
\vspace{3cm}
\begin{center}
Term paper

\vspace{1em}

\textbf{\Large{National Action Plans to implement the UN Guiding Principles on
Business and Human Rights}}

\vspace{1cm}

HAW Hamburg

Hamburg University of Applied Sciences

Faculty of Business \& Social Sciences

Department of Business

\vspace{1cm}

Foreign Trade/ International Management, B.Cs.

Academic research and Writing, Summer term 2020

Professor Dr. Christian Decker

Instructor Professor Michael Gille
\end{center}

\vfill

\begin{table}[h]
\begin{tabular}{ll}
Matriculation number:    & 2507884                                                                    \\ \\[-1em]
Name:                    & Tran, Thy                                                                  \\ \\[-1em]
Date and place of birth: & 24 January 1998, Vietnam                                                   \\ \\[-1em]
Telephone:               & 015736506729                                                               \\ \\[-1em]
Address:                 & Emil-Andressen-Straße 34 B, 22529 Hamburg                                  \\ \\[-1em]
Email:                   & Thy.Tran@haw-hamburg.de                                                    \\ \\[-1em]
                         &                                                                            \\ \\[-1em]
Date of submission:      & 
\end{tabular}
\end{table}


\thispagestyle{plain}
\textbf{Abstract}

Lorem ipsum dolor sit amet, consectetur adipiscing elit, sed do eiusmod tempor incididunt ut labore et dolore magna aliqua. Ut enim ad minim veniam, quis nostrud exercitation ullamco laboris nisi ut aliquip ex ea commodo consequat. Duis aute irure dolor in reprehenderit in voluptate velit esse cillum dolore eu fugiat nulla pariatur. Excepteur sint occaecat cupidatat non proident, sunt in culpa qui officia deserunt mollit anim id est laborum.

At vero eos et accusamus et iusto odio dignissimos ducimus qui blanditiis praesentium voluptatum deleniti atque corrupti quos dolores et quas molestias excepturi sint occaecati cupiditate non provident, similique sunt in culpa qui officia deserunt mollitia animi, id est laborum et dolorum fuga.

\vspace{2cm}
\textbf{Keywords}: crisis communication, crisis communication strategies, tourism industry, tourist organizations, coronavirus outbreak, Covid-19

\vspace{1cm}
\textbf{JEL classification}: I18, L98

\tableofcontents
\addcontentsline{toc}{chapter}{I. OUTLINE}
\listoffigures
\addcontentsline{toc}{chapter}{II. List of figures}
\listoftables
\addcontentsline{toc}{chapter}{III. List of tables}
\newpage
\thispagestyle{plain}
\linespread{1.25}
\textbf{IV. List of abbreviations}

\begin{flushleft}
\begin{table}[h]
\begin{tabular}{ll}
HRDD    & Human rights due diligence                                     \\ \\[-1em]
ICAR    & The International Corporate Accountability Roundtable          \\ \\[-1em]
ILO     & International Labour Organisation                              \\ \\[-1em]
NCP     & National Contact Points                                        \\ \\[-1em]
OECD    & Organisation for Economic Co-operation and Development         \\ \\[-1em]
UDHR    & Universal Declaration of Human Rights                          \\ \\[-1em]
UN      & United Nations                                                 \\ \\[-1em]
UNGP(s) & United Nations Guiding Principles on Business and Human Rights
\end{tabular}
\end{table}
\end{flushleft}

\addcontentsline{toc}{chapter}{IV. List of abbreviations}
%% \printglossaries
\mainmatter
\chapter{Introduction}
\vspace{6pt}
\section{Research questions}
\vspace{6pt}

Since the end of 2019, the world has faced with the biggest threat to the maintenance of international peace, security and economic stability, whose name is  The coronavirus (also being called Covid-19) pandemic caused by severe acute respiratory syndrome coronavirus 2 (SARS-CoV-2). The outbreak was identified in Wuhan, China in December 2019 and officially recognized as a pandemic on 11 March 2020. On 12 April 2020, more than 1.84 million cases of Covid-19 have been reported in 210 countries and territories worldwide, more than 113 000 deaths and more than 421 000 people have recovered. This coronavirus spreads primarily through droplets generated when an infected person coughs or sneezes, or through droplets of saliva or discharge from the nose. 

To counteract the fast contagiousness of this coronavirus, many countries worldwide have responded by implementing travel restrictions, quarantines, curfews, workplace hazard controls and facility closures. This coronavirus outbreak has brought considerable human suffering and economic disruption, including the direct disruption to global supply chains, weaker final demand for imported goods and services, and the regional declines in international tourism and business travel. 

Because of the unpredictable goings-on of this outbreak and its negative effects on tourism sector, the international as well as national tourist organisations must have urgent response in time in order to mitigate this global crisis. Consequently, having  crisis communication plans and strategies is undoubtedly necessary. Nonetheless, tourist organisations cannot implement the appropriated crisis communication strategies and reactions against the coronavirus without collaborating with other international organisations and national governments around the world.

This term paper will represent the influences of Covid-19 pandemic on tourism industry. The focus will be on analysing the crisis communication strategies and responses of international organisations, especially international tourist organisations and their accomplishments  in term of mitigating the severe global coronavirus crisis. The crisis communication response in the post-crisis will be represented at the end.

\vspace{12pt}
\section{Course of investigation}
\vspace{6pt}

Based upon the research question postulated in section 1.1, the negative impacts of coronavirus on tourism industry will be described in the second section. Following this is the analyse of crisis communication strategies and tools of tourist organisations in term of responding to the coronavirus outbreak. In the last section we will sum up the main crisis communication strategies as well as its outcome and envisage the solutions of tourism organisations in the last stage of coronavirus crisis to improve the tourism industry situation.

\vspace{12pt}
\chapter{Influences of coronavirus pandemic on tourism industry}
\vspace{6pt}
\section{Economics crisis in tourism industry}
\vspace{6pt}

The COVID-19 outbreak has brought our world to a standstill with unparalleled and unforeseen impact in our lives, our economies, our societies and our livelihoods and there are growing risks of a global recession and a massive loss of jobs. Based on the latest developments (quarantine measures, travel bans \& border closures in most of Europe, which represents 50\% of international tourism, and in many countries of the Americas, Africa and the Middle East), the evolutions in Asia and the Pacific and the patterns of previous crises (2003 SARS and 2009 global economic crisis), UNWTO estimates international tourist arrivals could decline by 20\% to 30\% in 2020.  This would translate into a loss of 300 to 450 US\$ billion in international tourism receipts (exports) – almost one third of the US\$ 1.5 trillion generated globally in the worst-case scenario \cite{unwto}.

\vspace{12pt}
\chapter{Crisis communication strategies of international tourism organisations}
\vspace{6pt}
\section{Implemented crisis communication strategies}
\vspace{6pt}
\subsection{Collaboration with worldwide organisations}
\vspace{6pt}

\vspace{12pt}
\subsection{Crisis communication tools and channels}
\vspace{6pt}

\vspace{12pt}
\section{The impacts of crisis communication strategies on the spread of coronavirus pandemic}
\vspace{6pt}

\vspace{12pt}
\chapter{Conclusion}
\vspace{6pt}
\section{Summary}
\vspace{6pt}

\vspace{12pt}
\section{Critical acclaim}
\vspace{6pt}

\vspace{12pt}
\section{Discussion}
\vspace{6pt}

%% \phantom{\cite{mauro_18AD},\cite{coombs_2019},\cite{glaesser_2006},\cite{hvass_2013},\cite{kerlin_nigel_2006},\cite{shaulova_biagi},\cite{serhan_2020}}
%% \printbibliography[title={V. List of references}]
\newpage
\newpage
\thispagestyle{plain}
\textbf{V. List of references}
\begin{flushleft}
		Baldwin, R., \& di Mauro, B. W. (Eds.). (18 March 2020). \textit{Mitigating the COVID Economic Crisis: Act fast and Do Whatever It Takes}. CEPR Press. Retrieved from https://voxeu.org/content/mitigating-covid-economic-crisis-act-fast-and-do-whatever-it-takes.

		Blackman, D. A., \& Ritchie, B. (2007, January). \textit{Tourism Crisis Management and Organizational Learning}. Retrieved from https://www.researchgate.net/publication/232985976

		Consult, M. (2020, March). \textit{Coronavirus: impact on the aviation industry worldwide}. (United States). Retrieved from https://www.statista.com/study/71610/coronavirus-impact-on-the-aviation-industry-worldwide/.

		Coombs, W., 2012. \textit{Ongoing Crisis Communication: Planning, Managing, Responding. 5th ed}. Thousand Oaks, Calif.: Sage Publications Ltd.

		Glaesser, D. (2006). \textit{Crisis management in the tourism industry}. Boston: Butterworth-Heinemann.

		Hvass, K. A. (2011, December). \textit{Tourism social media and crisis communication: An Erupting Trend}. (Denmark). Retrieved from https://www.researchgate.net/publication/293277102

		Kerlin.K. (2006). \textit{Expecting The Unexpected: Crisis Communication Preparedness In The Tourism Industry}. Retrieved from https://espace.curtin.edu.au/bitstream/handle/20.500.11937/14612/20803

		Koehl, D., (2011). \textit{Toolbox For Crisis Communications In Tourism- Checklist And Best Practices}. Madrid: World Tourism Organization.

		Mundi, L., \textit{World Ready, \& The World's Leading Network Independent Law Firms} | ESX Inc. (n.d.). (20 March 2020). COVID-19 Government Support Measures Report. Retrieved from https://www.lexmundi.com/lexmundi/COVID-19

		OECD. (2020, March 2). \textit{Interim Economic Assessment Coronavirus: The world economy at risk}. Retrieved from https://www.oecd.org/berlin/publikationen/Interim-Economic-Assessment-2-March-2020.pdf

		OWID; Roser, M., Ritchie, H., Ortiz-Ospina, E. (2020, April). \textit{The coronavirus disease (COVID-19) pandemic 2019-20}. Retrieved from https://www.statista.com/study/71007/the-coronavirus-disease-covid-19-outbreak/.

		Pearce.B. (7th April 2020). \textit{COVID-19 Wider economic impact from air transport collapse}. Retrieved from https://www.iata.org/en/iata-repository/publications/economic-reports/covid-19-wider-economic-impact-from-air-transport-collapse/

\newpage
\thispagestyle{plain}
		Ritchie. B., Dorrell. H., Miller.D., Miller.G. (January 2004). \textit{Crisis Communication and Recovery for the Tourism Industry}. DOI: 10.1300/J073v15n02\_11.

		Tourism Manifesto (Ed.). (2020, March 18). \textit{European Tourism Sector demands urgent supportive measures to reduce devastating impact of COVID-19}. Retrieved from https://www.hotrec.eu/european-tourism-sector-demands-urgent-supportive-measures-to-reduce-devastating-impact-of-covid-19/.

		Ulmer, R. R., Sellnow, T. L., \& Seeger, M. W. (2019). \textit{Effective crisis communication: moving from crisis to opportunity. Kindle edition (4th ed.)} Thousand Oaks: SAGE Publications.

		Who.int. (2020). \textit{Novel Coronavirus (2019-Ncov) Situation Reports}. Retrieved from https://www.who.int/emergencies/diseases/novel-coronavirus-2019/situation-reports.
\end{flushleft}

\addcontentsline{toc}{chapter}{V. List of references}
\newpage
\thispagestyle{plain}
\textbf{VI. Declaration of originality}

\vspace{6pt}

I hereby declare that this term paper and the work reported herein was composed by and originated entirely from me. Information derived from published and unpublished work of other has been acknowledged in the text and references are given in the list of references.

\begin{table}[h]
\begin{tabular*}{0.9\textwidth}{@{\extracolsep{\fill} } c c c r}
Place, Date                             & \phantom{o} & \phantom{o} & Signature            \\ \\[-1em]
\multicolumn{1}{l}{Hamburg, 2 May 2020} & \phantom{o} & \phantom{o} & \multicolumn{1}{l}{} \\ \\[2em]
 																				& \phantom{o} & \phantom{o} & Thy Tran
\end{tabular*}
\end{table}

\addcontentsline{toc}{chapter}{VI. Declaration of originality}
\end{document}
