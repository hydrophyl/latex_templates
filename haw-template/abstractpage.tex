\thispagestyle{plain}
\textbf{Abstract}

The attention of businesses and civil society groups for human rights due diligence has been greater since the adoption of the United Nations Protect, Respect and Remedy Framework in 2008 and its Guiding Principles on Business and Human Rights in 2011. Human rights due diligence plays the central role of the United Nations Guiding Principles on Business and Human Rights, which establishes the main parameters internationally for the considering corporate responsibility for human rights violations. This term paper will demonstrate the implementation of the United Nations Guiding Principles through national action plans by states and businesses and analyse the human rights due diligence process to identify and mitigate potential human rights risks for workers in its operations, supply chains and the services it uses.

\vspace{1cm}
\textbf{Keywords}: business and human rights, National Action Plans, UN Guiding Principles on Business and Human Rights, human rights due diligence.

\vspace{1cm}
\textbf{JEL classification}: K33, K38
