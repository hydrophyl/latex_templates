\documentclass[draft=false
              ,paper=a4
              ,twoside=false
              ,fontsize=12pt
              ,headsepline
              ,BCOR10mm
              ,DIV11
              ]{scrbook}
\usepackage[backend=biber,sorting=nty]{biblatex}
\addbibresource{citation.bib}
\usepackage[ngerman,english]{babel}
\usepackage[T1]{fontenc}
\usepackage[utf8]{inputenc}
\usepackage[top=2.5cm,left=3cm,right=2cm,bottom=2cm]{geometry}
\usepackage[parfill]{parskip}
\usepackage{hyperref}
\usepackage{glossaries} %% to do abbreviations
%% From HAW-Template
\usepackage{setspace}
\usepackage{listings}
\usepackage{graphicx} %% Important for including pictures
\usepackage{fancyhdr} %% for aligning header/footer
\pagestyle{fancy}
\hypersetup{
  colorlinks=true,
  linkcolor=black,
  filecolor=cyan,
  urlcolor=magenta,
  citecolor=red,
}
\linespread{1.25} %% Paragraphs 1.5 lineheight
\setlength{\parskip}{3pt}
%% \setmainfont{Arial}
\usepackage{uarial}
\renewcommand{\familydefault}{\sfdefault}

%% to set size of headings as normalsize 12pt
\usepackage{sectsty}
\chapterfont{\normalsize}
\sectionfont{\normalsize}
\subsectionfont{\normalsize}

%% to merge pdfpages
\usepackage{pdfpages}

%% to change space of headings to 0
\usepackage{titlesec}
\titlespacing*{\section}
{0pt}{0ex}{0ex}
\titlespacing*{\subsection}
{0pt}{0ex}{0ex}
\titlespacing*{\subsubsection}
{0pt}{0ex}{0ex}
\titleformat{\chapter}
  {\normalfont\normalsize\bfseries}{\thechapter}{1em}{}
\titlespacing*{\chapter}
{0pt}{0ex}{0ex}

%% Change title of \tableofcontents
\addto\captionsenglish{% Replace "english" with the language you use
  \renewcommand{\contentsname}%
    {I. OUTLINE}%
  \renewcommand{\listfigurename}%
  	{II. List of figures}
  \renewcommand{\listtablename}%
	{III. List of tables}
  \renewcommand{\refname}%
  	{V. List of references}
}

\begin{document}
\frontmatter

%% output title page and abstractpage
%% \includepdf[pages=-]{deckblatt.pdf}
\thispagestyle{empty}
\phantom{o}
\vspace{3cm}
\begin{center}
Term paper

\vspace{1em}

\textbf{\Large{National Action Plans to implement the UN Guiding Principles on
Business and Human Rights}}

\vspace{1cm}

HAW Hamburg

Hamburg University of Applied Sciences

Faculty of Business \& Social Sciences

Department of Business

\vspace{1cm}

Foreign Trade/ International Management, B.Cs.

Academic research and Writing, Summer term 2020

Professor Dr. Christian Decker

Instructor Professor Michael Gille
\end{center}

\vfill

\begin{table}[h]
\begin{tabular}{ll}
Matriculation number:    & 2507884                                                                    \\ \\[-1em]
Name:                    & Tran, Thy                                                                  \\ \\[-1em]
Date and place of birth: & 24 January 1998, Vietnam                                                   \\ \\[-1em]
Telephone:               & 015736506729                                                               \\ \\[-1em]
Address:                 & Emil-Andressen-Straße 34 B, 22529 Hamburg                                  \\ \\[-1em]
Email:                   & Thy.Tran@haw-hamburg.de                                                    \\ \\[-1em]
                         &                                                                            \\ \\[-1em]
Date of submission:      & 
\end{tabular}
\end{table}

\newpage
\thispagestyle{plain}
\textbf{Abstract}

In the globalization era, tourism industry is one of the main sectors that boost the thriving of worldwide economy. With the incessantly development in world trade, better means of transport and communications, international tourism has grown rapidly, as is reflected in the global trend of tourist arrivals and receipts. However, it also means that the infectious disease appearing in one country can spread faster which consequently lead to a global crisis when no countermeasure of international organizations and local governments is offered.

This term paper looks at the role of crisis communication strategies with regard to the tourism industry when facing with difficulties during coronavirus (COVID-19) outbreak. It considers the public response to coronavirus, the role of official organizations in avoiding the spread of coronavirus to protect community’s health.

\vspace{2cm}
\textbf{Keywords}: crisis communication, crisis communication strategies, tourism industry, tourist organizations, coronavirus outbreak, Covid-19

\vspace{1cm}
\textbf{JEL classification}: I18, L98


\tableofcontents
\addcontentsline{toc}{chapter}{I. OUTLINE}


\listoffigures
\addcontentsline{toc}{chapter}{II. List of figures}
\listoftables
\addcontentsline{toc}{chapter}{III. List of tables}

\newpage
\thispagestyle{plain}
\linespread{1.25}
\textbf{IV. List of abbreviations}

\begin{flushleft}
\begin{table}[h]
\begin{tabular}{ll}
HRDD    & Human rights due diligence                                     \\ \\[-1em]
ICAR    & The International Corporate Accountability Roundtable          \\ \\[-1em]
ILO     & International Labour Organisation                              \\ \\[-1em]
NCP     & National Contact Points                                        \\ \\[-1em]
OECD    & Organisation for Economic Co-operation and Development         \\ \\[-1em]
UDHR    & Universal Declaration of Human Rights                          \\ \\[-1em]
UN      & United Nations                                                 \\ \\[-1em]
UNGP(s) & United Nations Guiding Principles on Business and Human Rights
\end{tabular}
\end{table}
\end{flushleft}

\addcontentsline{toc}{chapter}{IV. List of abbreviations}

%% Begin document
\mainmatter
%% Chapter 1
\chapter{Introduction}
\vspace{6pt}
\section{Research problem}
\vspace{6pt}
The actions of business enterprises have a significant effect on people’s entitlement of their human rights either positively or negatively. On one hand, enterprises can enhance their workers’ working performance by delivering innovative technologies and services in production process or increasing salary. On the other hand, companies may be involved with negative human rights issues, for instance discrimination, sexual harassment, health \& safety, freedom of association and to form unions, rape, torture, freedom of expression, privacy, poverty, food and water, education and housing. The United Nations Guiding Principles on Business and Human Rights (UNGPs)  which were adopted by consensus of the United Nations Human Rights Council (UNHRC) in June 2011 address this need and offer the first international reference framework and human rights in the context of business, clearly defining the duties and responsibilities of all players in three-pillar model known as the “Protect, Respect and Remedy” Framework including 31 principles. Although protecting human rights is the main duty of national governments, companies also have a responsibility to respect human rights and follow the national action plans in this field.

National Action Plans (NAPs) were first developed in 2011 by the European Union, the Council of Europe, the Organisation of American States, the G7, the G20, national human rights institutions and business associations in order to support the implementation of the UN Guiding Principles on Business and Human Rights. The UN Working Group on business \& human rights encourages governments to adopt NAPs on business \& human rights as "an important means to promote the implementation of the UNGPs", specifically of the state duty to protect human rights. In 2019, over 50 countries have produced a national action plan or have been in the process of developing a NAP or have committed to developing one.

The implementation of UN Guiding Principles in business enterprises or in the state’s territory should be analysed to support the design of measures within a national action plan. This analyse comprises assessing to what extent businesses have committed to respecting human rights and carrying out human rights due diligence. The establishment of human rights due diligence process can help enterprises to reduce or counterbalance adverse impacts on human rights. In other word, implementing human rights due diligence is the responsibility of enterprises to manage adverse human rights impacts. This is also the main focus of UN Guiding Principles on Business and Human Rights with five out of 31 principles.

\vspace{12pt}
\section{Research method}\label{research_method}
\vspace{6pt}
This term paper aims to analyse steps of national action plans in the process of implementing the UN Guiding Principles on Business and Human Rights. The focus of this term paper will be on depicting the integration of human rights due diligence into policies and procedures in companies. The purpose  of this paper is to answer the following questions:

\begin{itemize}
\item{How are the UN guiding principles through national action plans implemented?}
\item{How can human rights due diligence be integrated in enterprises?}
\end{itemize}

\vspace{12pt}
\section{Course of investigation}
\vspace{6pt}
Based upon the research question postulated in section \ref{research_method}, the function and development of a national action plan will be described in the second chapter. In the third chapter, UN Guiding Principles on Business and Human and the implementation of the UN Guiding Principles in national action plans will be illustrated in the first and second sub-section respectively. The role of human rights due diligence in enterprises as well as in supporting the UN Guiding Principles will be presented in the fourth chapter. The conclusion will be placed at the end of the main body in this term paper.




%% Chapter 2
\vspace{12pt}
\chapter{National Action Plans}
\vspace{6pt}
\section{Value of national action plans on Business and Human Rights}
\vspace{6pt}

\vspace{12pt}
\section{Developing a national action plan }
\vspace{6pt}






% Chapter 3
\vspace{12pt}
\chapter{UN Guiding Principles on Business and Human Rights}
\vspace{6pt}
\section{The role of UN Guiding Principles on Business and Human Rights}
\vspace{6pt}

\vspace{12pt}
\section{Three-pillar Framework}
\vspace{6pt}

\vspace{12pt}
\section{National Action Plans to implement UN Guiding Principles}
\vspace{6pt}






% Chapter 4
\vspace{12pt}
\chapter{Human rights due diligence}
\vspace{6pt}
\section{Human rights due diligence in UN Guiding Principle}
\vspace{6pt}

\vspace{12pt}
\section{Integration of human rights due diligence }
\vspace{6pt}









% Chapter 5
\vspace{12pt}
\chapter{Conclusion}
\vspace{6pt}
\section{Summary}
\vspace{6pt}

\vspace{12pt}
\section{Critical acclaim}
\vspace{6pt}

\vspace{12pt}
\section{Outlook}
\vspace{6pt}






%% Bibliography
\newpage
\textbf{V. List of references}
\begin{flushleft}
Baldwin, R., \& di Mauro, B. W. (Eds.). (18 March 2020). Mitigating the COVID Economic Crisis: Act fast and Do Whatever It Takes. CEPR Press. Retrieved from https://voxeu.org/content/mitigating-covid-economic-crisis-act-fast-and-do-whatever-it-takes.

Blackman, D. A., \& Ritchie, B. (2007, January). Tourism Crisis Management and Organizational Learning. Retrieved from https://www.researchgate.net/publication/232985976

Consult, M. (2020, March). Coronavirus: impact on the aviation industry worldwide. (United States). Retrieved from https://www.statista.com/study/71610/coronavirus-impact-on-the-aviation-industry-worldwide/.

Coombs, W., 2012. Ongoing Crisis Communication: Planning, Managing, Responding. 5th ed. Thousand Oaks, Calif.: Sage Publications Ltd.

Glaesser, D. (2006). Crisis management in the tourism industry. Boston: Butterworth-Heinemann.

Hvass, K. A. (2011, December). Tourism social media and crisis communication: An Erupting Trend. (Denmark). Retrieved from https://www.researchgate.net/publication/293277102

Kerlin.K. (2006). Expecting The Unexpected: Crisis Communication Preparedness In The Tourism Industry. Retrieved from https://espace.curtin.edu.au/bitstream/handle/20.500.11937/14612/20803

Koehl, D., (2011). Toolbox For Crisis Communications In Tourism- Checklist And Best Practices. Madrid: World Tourism Organization.

Mundi, L., World Ready, \& The World's Leading Network Independent Law Firms | ESX Inc. (n.d.). (20 March 2020). COVID-19 Government Support Measures Report. Retrieved from https://www.lexmundi.com/lexmundi/COVID-19

OECD. (2020, March 2). Interim Economic Assessment Coronavirus: The world economy at risk. Retrieved from https://www.oecd.org/berlin/publikationen/Interim-Economic-Assessment-2-March-2020.pdf

OWID; Roser, M., Ritchie, H., Ortiz-Ospina, E. (2020, April). The coronavirus disease (COVID-19) pandemic 2019-20. Retrieved from https://www.statista.com/study/71007/the-coronavirus-disease-covid-19-outbreak/.

Pearce.B. (7th April 2020). COVID-19 Wider economic impact from air transport collapse. Retrieved from https://www.iata.org/en/iata-repository/publications/economic-reports/covid-19-wider-economic-impact-from-air-transport-collapse/.

Ritchie. B., Dorrell. H., Miller.D., Miller.G. (January 2004). Crisis Communication and Recovery for the Tourism Industry. DOI: 10.1300/J073v15n02\_11.

Tourism Manifesto (Ed.). (2020, March 18). European Tourism Sector demands urgent supportive measures to reduce devastating impact of COVID-19. Retrieved from https://www.hotrec.eu/european-tourism-sector-demands-urgent-supportive-measures-to-reduce-devastating-impact-of-covid-19/.

Ulmer, R. R., Sellnow, T. L., \& Seeger, M. W. (2019). Effective crisis communication: moving from crisis to opportunity. Kindle edition (4th ed.) Thousand Oaks: SAGE Publications.

Who.int. (2020). Novel Coronavirus (2019-Ncov) Situation Reports. Retrieved from https://www.who.int/emergencies/diseases/novel-coronavirus-2019/situation-reports.
\end{flushleft}

\addcontentsline{toc}{chapter}{V. List of references}
\newpage
\thispagestyle{plain}
\textbf{VI. Declaration of originality}

\vspace{6pt}

I hereby declare that this term paper and the work reported herein was composed by and originated entirely from me. Information derived from published and unpublished work of other has been acknowledged in the text and references are given in the list of references.

\begin{table}[h]
\begin{tabular*}{0.9\textwidth}{@{\extracolsep{\fill} } c c c r}
Place, Date                             & \phantom{o} & \phantom{o} & Signature            \\ \\[-1em]
\multicolumn{1}{l}{Hamburg, 2 May 2020} & \phantom{o} & \phantom{o} & \multicolumn{1}{l}{} \\ \\[2em]
 																				& \phantom{o} & \phantom{o} & Thy Tran
\end{tabular*}
\end{table}

\addcontentsline{toc}{chapter}{VI. Declaration of originality}
\end{document}
